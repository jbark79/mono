\documentclass[11pt]{article}
\usepackage[utf8]{inputenc}
\usepackage{graphicx}
\usepackage{xcolor}
\usepackage[ngerman]{babel}
\usepackage{enumitem}
\usepackage{array}
\usepackage{amsmath}
\usepackage{amsmath}
\usepackage{amsthm}
\usepackage{amssymb}
\usepackage{xcolor}
\definecolor{commentgreen}{RGB}{2,112,10}
\definecolor{eminence}{RGB}{108,48,130}
\definecolor{weborange}{RGB}{255,165,0}
\definecolor{frenchplum}{RGB}{129,20,83}

\newtheorem{theorem}{Theorem}
\newtheorem{proposition}{Proposition}
\newtheorem*{lemma}{Lemma}

\setlength{\parindent}{0em}

\title{\vspace{-4.7cm}EA und RSA}
\author{Barkoczi, Weskamp}
\date{}

\begin{document}
\maketitle
\section{Rechnen mit Modulo}
Mit Hilfe der Modulorechnung k\"onnen wir Gemeinsamkeiten von Zahlen unter bestimmten Moduli untersuchen. Um kennzuzeichen,
dass Zahlen unter einem bestimmten Modul \textit{kongruent} sind, benutzen wir das Symbol $\equiv$. Seien $a,b,c \in \mathbb{Z}$, dann
gilt 
\[
    a \equiv b \mod c \iff c \mid a-b.
\]
Insbesondere gilt also 
\[
    a \equiv b \mod c \iff a-b = k \cdot c,
\]
f\"ur ein $k \in \mathbb{Z}$.\\

Beispielsweise ist $12 \equiv 8 \mod 2$, da $2$ ein Teiler von $12-8=4$ ist.
Aus der Definition folgt ebenfalls, dass zwei Zahlen $x,y$ kongruent $\mod z$ sind, wenn 
$x$ und $y$ denselben Rest nach einer Division durch $z$ haben.

\section{Euklidischer Algorithmus}
Der euklidische Algorithmus ist eine effiziente Methode um den gr\"o{\ss}ten gemeinsamen
Teiler zweier Zahlen $a,b$ zu bestimmen. Er funktioniert wie folgt:
\begin{enumerate}
    \item Man startet mit $a_1 = a$ und $b_1 = b$ und schreibt eine Gleichung in der Form \[ a_1 = b_1 \cdot x_1 + r_0, \]
        wobei $x_1$ gr\"o{\ss}tm\"oglich ist.
    \item Jede Gleichung f\"uhrt zu einer Neuen, bei der jedoch das vorherige $b_i$ zum neuen $a_{i+1}$ und das vorherige
        $r_{i}$ zum neuen $b_{i+1}$ wird.
    \item Dies f\"uhrt man solange durch, bis $r_i = 0$ gilt, und der Rest $r_{i-1}$ ist der gesuchte ggT.
\end{enumerate}
Ein konkretes Beispiel mit $a=126, b=72$:\\
\begin{align*}
    126 &= 78 \cdot 1 + 48\\
    78 &= 48  \cdot 1 + 30\\
    48 &= 30  \cdot 1 + 18\\
    30 &= 18  \cdot 1 + 12\\
    18 &= 12  \cdot 1 + 6\\
    12 &= 6   \cdot 2 + 0.\\
\end{align*}
Wir sehen, der vorletzte Rest, in dem Fall $r_5$ ist unser gesuchter ggT.

\pagebreak
\section{Erweiterter euklidischer Algorithmus}
Der erweiterte euklidische Algorithmus liefert uns eine Methode, bei gegebenen $b,c$ Gleichungen der Form
\[
    1 = bx + cy
\]
nach $(x,y)$ zu l\"osen. Diese Gleichung kann man auch wie folgt umstellen
\begin{align*}
    & 1 = bx + cy \\ 
    \iff & 1 - bx = cy \\ 
    \iff & c \mid 1 - bx \\ 
    \iff & c \mid bx - 1. \\ 
\end{align*}
Dies ist jedoch wie in Sektion 1. erw\"ahnt equivalent zu $bx \equiv 1 \mod c$. Man sieht also eine Verbindung zwischen modularer Arithmetik
und dem erweiterten euklidischen Algorithmus, bzw. dass sich Gleichungen in der Form mit Hilfe des erweiterten euklidischen Algorithmus l\"osen 
lassen.

Der Algorithmus an sich l\"asst sich am besten anhand eines konkreten Beispiels erkl\"aren. Also sei $b=1260$ und $c=31$
\begin{enumerate}
    \item Man wendet erst den normalen euklidischen Algorithmus an. Es folgen die Gleichungen:
    \begin{align*}
        1260 &= 31 \cdot 40 + 20\\ 
        31 &= 20 \cdot 1 + 11\\ 
        20 &= 11 \cdot 1 + 9\\ 
        11 &= 9 \cdot 1 + 2\\ 
        9 &= 2 \cdot 4 + 1\\ 
        2 &= 1 \cdot 2 + 0\\ 
    \end{align*}
    Der ggT ist also $1$. 
    \pagebreak
    \item Nun stellen wir alle Gleichungen au{\ss}er die Letzte jeweils nach dem Rest um.
    \begin{align}
        20 &= 1260 + 31 \cdot (-40)\\ 
        11 &= 31 + 20 \cdot (-1)\\
        9 &= 20 + 11 \cdot (-1)\\
        2 &= 11 + 9 \cdot (-1)\\
        1 &= 9 + 2 \cdot (-4)
    \end{align}
    Zur Erinnerung: unser Ziel ist es, die Gleichung $1 = 1260 \cdot x + 31 \cdot y$ nach $(x,y)$ aufzul\"osen.
    Die 5. Gleichung scheint schon mal in die richtige Richtung zu gehen.
    \item Der n\"achste Schritt ist, uns von unten nach oben durch die Gleichungen zu arbeiten, in dem wir die Oberen in die Untere 
        einsetzen. Wir fangen also an, die 4. in die 5. zu setzen, und erhalten
        \begin{align*}
            1 &= 9 + 2 \cdot (-4)\\
              &= 9 + (11+9\cdot(-1)) \cdot (-4),
        \end{align*}
        Nun fassen wir zusammen und erhalten
        \begin{align*}
            1 &= 9 + (11+9\cdot(-1)) \cdot (-4)\\
              &= 9 \cdot 5 + 11 \cdot (-4).
        \end{align*}
        Dieses Einf\"ugen und Zusammenfassen machen wir nun solange bis wir an die gew\"unschte Form kommen.
        \begin{align*}
            1 &= 9 \cdot 5 + 11 \cdot (-4)\\ 
              &= (20+11\cdot(-1))\cdot 5 + 11 \cdot (-4)\\
              &= 20 \cdot 5 + 11 \cdot (-9)\\
              ~\\
              &= 20 \cdot 5 + (31+20\cdot(-1)) \cdot (-9)\\
              &= 20 \cdot 14 + 31 \cdot (-9)\\
              ~\\
              &= (1260+31\cdot(-40)) \cdot 14 + 31 \cdot (-9)\\
              &= 1260 \cdot 14 + 31 \cdot (-569)\\
        \end{align*}
        Und wir sehen unsere L\"osung f\"ur (x,y) ist (14,-569). Dies ist \"ubrigens nur \textit{eine} L\"osung von \textit{unendlich} vielen L\"osungen. 
        Weitere kann man durch die Addition der Gleichung mit $1260\cdot31 - 31\cdot 1260$ und geschicktem Umformen generieren
        \begin{align*}
            1 &= 1260 \cdot 14 + 31 \cdot (-569)\\
            1 &= 1260 \cdot 14 + 31 \cdot (-569) + 1260\cdot31 - 31\cdot1260\\
            1 &= 1260 \cdot (14-31) + 31 \cdot (-569+1260)\\
        \end{align*}
        Das Paar $(x,y) = (14-31,-569+1260) = (-17, 691)$ ist also beispielsweise eine weitere L\"osung.
\end{enumerate}

\section{RSA - Schl\"usselkonstruktion}
Ein RSA-Schl\"usselpaar besteht aus zwei Teilen. Dem \"offentlichen Schl\"ussel den man ohne Probleme frei in Umlauf geben kann,
und dem privaten bzw. geheimen Schl\"ussel, welcher unter keinen Umst\"anden weitergegeben werden darf. Die Schl\"usselkonstruktion
l\"auft wie folgt ab:
\begin{enumerate}
    \item 
        Man w\"ahlt zwei ausreichend gro{\ss}e (in der Regel 1000-stellig) Primzahlen $p,q$ und bildet das Produkt $N := pq$.
        Dann sucht man eine ebenfalls ausreichend gro{\ss}e ganze Zahl $e$, die teilerfremd zu $\varphi(N)$ ist, wobei $\varphi$ die eulersche 
        Phi-Funktion ist. Das Paar $(N,e)$ bildet unseren \"offentlichen RSA-Schl\"ussel.

    \item
        Da $\gcd(e,\varphi(N)) = 1$ gilt, ist die Gleichung \[ 1 = e \cdot d + \varphi(N) \cdot y \] nach (d,y) l\"osbar und insbesondere 
        interessiert uns eine positive L\"osung f\"ur e. Dies ist nun mit Hilfe des erweiterten euklidischen Algorithmus leicht machbar.
        Unseren privaten RSA-Schl\"ussel bildet somit das Paar $(N,d)$. 
\end{enumerate}
Eine Anmerkung ist, dass man nur mit dem Wissen von $\varphi(N)$ leicht an $d$ rankommt.
Um $\varphi(N)$ jedoch zu bestimmen, braucht man die Primfaktorisierung von $N$. Da es keinen effizienten Algorithmus f\"ur 
die Primfaktorisierung von Zahlen in dieser Gr\"o{\ss}enordnung gibt, ist es also praktisch nicht m\"oglich, mit dem alleinigem Wissen 
\"uber $N$ und $e$, also dem \"offentlichen RSA-Schl\"ussel, $d$ zu finden. Dies macht das RSA-Verfahren robust.

\subsection{Beispiel}
Im folgenden werden wir beispielshalber ein RSA-Schl\"usselpaar konstruieren. Wir w\"ahlen $p=19$ und $q=71$.
\footnote{Normalerweise w\"urde man nat\"urlich deutlich gr\"o{\ss}ere Zahlen w\"ahlen}
Das Primzahlprodukt ist $N=19\cdot71=1349$ und $\varphi(N) = 18 \cdot 70 = 1260$. F\"ur $e$ w\"ahlen wir 31 und 
da $\gcd(31,1260) = 1$ gilt, bildet dies einen wohlgeformten \"offentlichen RSA-Schl\"ussel $(1349,31)$.

Um unseren privaten RSA-Schl\"ussel zu bilden, l\"osen wir die Gleichung \[ 1 = 1260 \cdot x + 31 \cdot e \] nach $(x,e)$, wobei wir eine
positive L\"osung f\"ur $e$ wollen. Aber genau diese Gleichung haben wir in Sektion 3 gel\"ost und haben als L\"osung unter anderem 
$(x,e) = (-17,691)$ erhalten. Somit lautet also unser privater RSA-Schl\"ussel $(1349,691)$.

\section{RSA - Verschlüsselung und Entschlüsselung}
Gegeben ist das folgende Lemma.
\begin{lemma}
    Sei $x \in \mathbb{Z}$, $N$ = $pq$ mit $p,q \in \mathbb{P}$ und $e,d \in \mathbb{Z}$, dann gilt
    \[e \cdot d \equiv 1 \mod \varphi(N) \implies x^{ed} \equiv x \mod N\]
    f"ur fast beliebige $x \in \mathbb{N}$.
\end{lemma}
\begin{proof}
    Siehe Gruppe 3.
\end{proof}
Wenn wir zeigen k\"onnen, dass $e \cdot d \equiv 1 \mod \varphi(N)$ gilt, dann wissen wir dank des Lemmas, dass
$x^{ed} \equiv x \mod \varphi(N)$ gilt. Jedoch ist $e \cdot d \equiv 1 \mod \varphi(N)$  equivalent zu $\varphi(N) 
\mid e\cdot d -1$, bzw. $e\cdot d - k \cdot \varphi(N) = 1$ f\"ur ein $k \in \mathbb{Z}$. Aber das dies gilt haben wir 
ja schon in unserer Schl\"usselkonstruktion gezeigt. Daher gilt nach dem Lemma $x^{ed} \equiv x \mod N$, was wir jetzt anwenden werden.
\subsection{Verschl\"usselung und Entschl\"usselung}
Sei $(N,e)$ ein beliebiger \"offentlicher RSA-Schl\"ussel und 
sei $x \in \mathbb{Z}$ unsere Geheimnachricht, dann definieren wir $y := x^e \mod N$ als unsere verschl\"usselte Nachricht.

Sei $(N,d)$ nun der zu $(N,e)$ passende private RSA-Schl\"ussel. Dann gilt \[y^d \equiv (x^e)^d \equiv x^{ed} \equiv x \mod N.\]
Und somit k\"onnen wir erfolgreich Ver- und Entschl\"usseln.

\section{RSA - Square-and-Multiply Algorithmus}
Der Square-and-Multiply Algorithmus wird genutzt um bei gegebenem $a,b,c$ Gleichungen der Form $a^b \equiv x \mod c$ effizient
nach $x$ zu l\"osen. 

Dies ist wichtig, da bei dem RSA-Verfahren mit sehr gro{\ss}en Exponenten gerechnet wird. Am besten l\"asst sich
der Algorithmus anhand des folgenden Beispiels demonstrieren: 
Sei $a=269, b=159, c=1251$. Den Exponenten schreiben wir nun in Bin"arform $b \equiv (10011111)_2$. Dann erstellen wir
eine Tabelle mit den folgenden Spalten
\begin{itemize}
    \item $i$: Dies ist eine Laufvariable.\\
    \item $b_i$: Hier schreiben wir die $b_i$'te Ziffer der Bin"arform von $b$ (von rechts aus).\\
    \item $a$: Hier speichern wir den aktuellen Wert $a_i$, wobei $a_i \equiv a^{2^i} \mod c$\\
    \item $z$: Hier speichern wir den aktuellen Wert $z_i$, wobei $z_i \equiv z_{i-1} \cdot a_i \mod c$ falls $b_i = 1$, und sonst 
        "ubernehmen wir $z_{i-1}$. Dies hat den Effekt, das $z_i \equiv a^{(b_i, b_{i-1}, \ldots, b_0)_2} \mod c$ speichert. Somit gilt,
        wenn $n$ die Anzahl der Ziffern von $(b)_2$ ist, dass $z_n \equiv a^b \mod c$ gilt.
    \item Berechnet: Ein Hinweis, welcher Rechnung $z_i$ entspricht.
\end{itemize}
\begin{center}
    \begin{tabular}{| m{0.2cm} | m{0.2cm}| m{4cm} | m{5cm} | m{3cm} |}
    \hline
        $i$ & $b_i$ & $a$ & $z$ & Berechnet\\
    \hline
    $0$ & $1$ & $269 \equiv 269 \mod 1251$ & $269 \equiv 269 \mod 1251$ & $269^1 \mod 1251$\\
    \hline                                    
    $1$ & $1$ & $269^2 \equiv 1054 \mod 1251$ & $269 \cdot 1054 \equiv 800 \mod 1251$& $269^3 \mod 1251$\\
    \hline                                    
    $2$ & $1$ & $1054^2 \equiv 28 \mod 1251$ & $800 \cdot 28 \equiv 1133 \mod 1251$& $269^7 \mod 1251$\\
    \hline                                    
        $3$ & $1$ & $28^2 \equiv 784 \mod 1251$ & $1133 \cdot 784 \equiv 62 \mod 1251$& $269^{15} \mod 1251$\\
    \hline                                    
        $4$ & $1$ & $784^2 \equiv 415 \mod 1251$ & $62 \cdot 415 \equiv 710 \mod 1251$& $269^{31} \mod 1251$\\
    \hline                                    
        $5$ & $0$ & $415^2 \equiv 838 \mod 1251$ & $710$& $269^{31} \mod 1251$\\
    \hline                                    
        $6$ & $0$ & $838^2 \equiv 433 \mod 1251$ & $710$& $269^{31} \mod 1251$\\
    \hline                                    
        $7$ & $1$ & $433^2 \equiv 1090 \mod 1251$ & $710 \cdot 1090 \equiv 782 \mod 1251$& $269^{159} \mod 1251$\\
    \hline
\end{tabular}
\end{center}
Man sieht, dass man mit Hilfe des Algorithmus die Rechnung in einer zu $b$ logarithmischen Anzahl an Schritten l"osen kann.
Das hei{\ss}t, dass man beispielsweise statt $4294967296$ Schritten mit dem naiven Weg, nur noch etwa $32$ Schritte mit Square-and-Multiply braucht!

\end{document}
